% pulp-Dp.tex
\begin{hcarentry}[new]{pulp}
\report{Daniel Wagner}%05/15
\status{Not yet released}
\participants{Daniel Wagner, Michael Greenberg}
\makeheader

Anybody who has used \LaTeX\ knows that it is a fantastic tool for
typesetting; but its error reporting leaves much to be desired. Even simple
documents that use a handful of packages can produce hundreds of lines of
uninteresting output on a successful run. Picking out the parts that require
action is a serious chore, and locating the right part of the document
source to change can be tiresome when there are many files.

Pulp is a parser for \LaTeX\ log files with a small but expressive
configuration language for identifying which messages are of interest. A
typical run of pulp after successfully building a document produces no
output; this makes it very easy to spot when something has gone wrong. Next
time you want to produce a great paper, process your log with pulp!

\paragraph*{Features}
\begin{compactitem}
	\item \LaTeX\ log parser with special-case support for many popular
		packages and classes
	\item Expressive configuration language
		\begin{compactitem}
			\item Filter out document-specific unimportance
			\item Increase verbosity as the document nears completion
		\end{compactitem}
	\item Uniform error reporting format with file and line information
	\item Instructions for use with latexmk
	\item Rudimentary Windows support
\end{compactitem}

% What's following are suggestions for the content of an entry.
%
% (WHAT IS IT?)
%
% (WHAT IS ITS STATUS? / WHAT HAS HAPPENED SINCE LAST TIME?)
%
% (CAN OTHERS GET IT?)
%
% (WHAT ARE THE IMMEDIATE PLANS?)

\FurtherReading
  \url{http://github.com/dmwit/pulp}
\end{hcarentry}
